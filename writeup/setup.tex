\section{Experimental Setup}
\label{sec:setup}

\subsection{MSI Dataset}
\paragraph{Mass spectrometry imaging with ion-mobility}
Mass spectrometry measures ions that are derived from the molecules present in a complex biological sample.
These spectra can be acquired at each location (pixel) of a heterogeneous sample, allowing for collection of spatially
resolved mass spectra.
This mode of analysis is known as \textit{mass spectrometry imaging (MSI)}.
The addition of \textit{ion-mobility separation (IMS)} to MSI adds another dimension, drift time
The combination of IMS with MSI is finding increasing applications in the study of disease diagnostics, plant
engineering, and microbial interactions. Unfortunately, the scale of MSI data and complexity of analysis presents a significant challenge to
scientists: a single 2D-image may be many gigabytes and comparison of multiple images is beyond the capabilities
available to many scientists. The addition of IMS exacerbates these problems.

Dimensionality reduction techniques can help reduce MSI datasets to more
amenable sizes.  Typical approaches for dimensionality reduction include PCA
and NMF, but interpretation of the results is difficult because the components
extracted via these methods are typically combinations of many hundreds or
thousands of features in the original data.  CX decompositions circumvent this
problem by identifying small numbers of columns in the original data that
reliably explain a large portion of the variation in the data.  This
facilitates rapidly pinpointing important ions and locations in MSI
applications.

In this paper, we analyze one of the largest (1TB sized) mass-spec imaging
datasets in the field. The sheer size of this dataset has previously made
complex analytics intractable. This paper presents first-time science results
from the successful application of CX to TB-sized data.

\subsection{Empirical Orthogonal Function analysis of ocean temperatures}
The most widely used tool for extracting important patterns from the
measurements of atmospheric and oceanic variables is the Empirical Orthogonal
Function (EOF) technique. EOFs are popular because of their simplicity and
their ability to reduce the dimensionality of large nonlinear, high-dimensional
systems into fewer dimensions while preserving the most important patterns of
variations in the measurements. Mathematically, EOFs are exactly PCA decompositions.

Traditionally, ocean data EOFs have been calculated for surface fields due to
the complications arising from dealing with the amount of data in the full 3D fields. 
However, the EOF modes extracted from such data only
contain projections of complex 3D patterns involving deeper layers of the
ocean on to the surface. A better understanding of the dynamics of large-scale
modes of variability in the ocean may be extracted from full 3D EOFs including
many layers of the ocean. 

We applied our PCA implementation in Spark to the analysis of the 
ocean temperature fields from 31 years worth of ocean temperature measurements taken 
at 6 hours increments over a global 360-by-720-by-40 global grid, extracted from
the Climate Forecast System Reanalysis Product~\cite{Saha:2010ji}. To the best of our knowledge, this is the
first published work where a 3D climate field of this magnitude has been subjected to 
EOF analysis.

\subsection{Platforms}

  \begin{table*}
    \begin{center}
    \begin{tabular}{| l | c | c | c | c | c | c | c |}
    \toprule
    \textbf{Platform} & \textbf{Total Cores} & \textbf{Core Frequency} & \textbf{Interconnect} & \textbf{DRAM} & \textbf{SSDs} \\
    \midrule
    Amazon EC2 \texttt{r3.8xlarge} & 960 (32 per-node) & 2.5 GHz & 10 Gigabit Ethernet & 244 GiB & 2 x 320 GB \\
    \midrule
    Cray XC40 & 960 (32 per-node) & 2.3 GHz & Cray Aries~\cite{alverson2012cray,craycascadesc12} & 128 GiB & None \\
    \midrule
    Experimental Cray cluster & 960 (24 per-node) & 2.5 GHz & Cray Aries~\cite{alverson2012cray,craycascadesc12} & 126 GiB & 1 x 800 GB \\
    \bottomrule
    \end{tabular}
    \end{center}
    \caption{Specifications of the three hardware platforms used in these performance experiments.}
    \label{tab:hwspecs}
  \end{table*}

 In order to assess the relative performance of CX and PCA matrix factorization on various hardware, we choose the following contemporary platforms:
 \begin{itemize}
 \item a Cray\textregistered~XC40\textsuperscript{\tiny\texttrademark}
 system~\cite{alverson2012cray,craycascadesc12},
 \item an experimental Cray cluster, and
 \item an Amazon EC2 \texttt{r3.8xlarge} cluster.
 \end{itemize}

 For all platforms, we sized the Spark job to use 960 executor cores (except as
 otherwise noted).  Table~\ref{tab:hwspecs} shows the full specifications of
 the three platforms. Note that these are state-of-the-art configurations in
 datacenters and high performance computing centers. 

