\section{Conclusions}
\label{sec:conclusion}

Matrix factorizations are an important and ubiquitous class of linear algebra
computations. In this work, we have developed a distributed
implementation of the CX algorithm and quantified its performance relative to
an optimized parallelized C baseline, and developed a similarly distributed
implementation of the PCA algorithm. We used Spark, a contemporary data
analytics framework, to develop and deploy our CX and PCA implementations, and
demonstrated that they scale up to 960 cores. We
evaluated our Spark CX implementation on HPC and EC2 class hardware and found
that faster interconnects enable numerically intensive computations to run
more efficiently on HPC systems. The CX implementation was used
to analyze a TB-sized mass-spec imaging dataset; the resulting ion and
spatial patterns obtained from the analysis are providing biologists with novel
insights on complex molecular mechanisms in cells. The PCA implementation was used
to analyze a 2 TB-sized ocean temperature dataset, and has enable climate scientists
to see structures in the full 3D field that were obfuscated in the 2D
projections they previously dealt with for computational convenience. Further
examination of the performance gap of the Spark CX implementation from the
optimized C implementation, and the attribution to various components of the
Spark stack will be conducted in the near future. Likewise, future work
examining the gap between our PCA implementation and more traditional HPC codes
on the platforms considered here will be conducted in the near future.


