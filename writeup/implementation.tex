
\section{High Performance Implementation}
\label{sec:implementation}

We compare implementations of the PCA and randomized SVD algorithms on 
an XC40 Cray platform and the EC2 platform.  The standard
numerical linear algebra libraries do not provide CX implementations, so we
produce a highly optimized, close-to-the-metal C implementation that focuses on
obtaining peak efficiency from conventional multi-core CPU chipsets and extend
it to multiple nodes.  Secondly, we implement the CX and PCA methods in Spark,
an emerging standard for parallel data analytics frameworks. 

\subsection {C+MPI CX implementation}
\label{sxn:single_node_opt}

We began by optimizing the steps in Algorithm~\ref{alg:cx}. Our first optimization is to perform $C$ = $AB$, 
followed by $\mathrm{Res} = A^TC$. This reduces the run-time complexity from
O({\it{n*(nsm)}}) to O({\it{k*(nsm)}}). Note that we do not
explicitly compute (or store) the transpose of $A$. We are able to exploit SIMD in the
computation of $\mathrm{Res}$ to speed up the computation by a factor equal to
the SIMD width (the number of simultaneous operations that can be performed).
Similarly, the computation of $A^TC$ is sped-up by a factor equal to the SIMD width.

We achieve load balancing in our computation of $Res$ by dividing
the rows such that each node operates on the same number of non zeros.  We
perform this partitioning using a two step process. In the first step, we
equally divide the number of rows, and each node reads in the corresponding
part of the matrix, and computes the number of non-zeros read. This is followed
by a redistribution step, where each node computes and distributes the relevant
rows amongst the cores in order to better balance the computation. We also use
cache blocking to efficiently distribute the matrix $A$ so that most of the
fetching comes from the last level cache; this mitigates the tendency of the
computation to be bound by the available memory bandwidth when $n$ is large, so
each processor requires more memory to do its assigned tasks. The amount of
data transferred between nodes is only a small fraction of the total input size
(measured $<$ 0.01\%), and this step is only performed once during the
execution of the algorithm. Each node computes the local resultant matrix
$Res$, which is then reduced globally to compute the final matrix. Note that
$Res$ consists of {\it{n}} $\times$ {\it{k}} elements, which occupies a few MBs
even for our 1 TB datset (recall {\it{m}} $\gg$ {\it{n}}).  A similar work
division scheme is used to compute $AQ$ (Step 7) in a distributed fashion.
Because of the small size of the matrices involved, the final two steps
(\textsc{ThinSVD} and a matrix multiplication) are performed on a single node.

Given the small size of the matrix involved in the QR step,
({\it{n}} $\times$ {\it{k}}), the QR is performed on a single-node, but
parallelized to exploit the multiple cores available, with the resultant matrix
being broadcast to all other nodes at the end of the computation.

Our final optimization is to account for multi-socket architectures, wherein
each socket has its own compute and memory resources. All cross-socket traffic
passes through a cross-socket link, which has lower bandwidth than access to
local DRAM/caches. Hence, we need to optimize for the amount of data
transferred between sockets. We divide the allocation equally between the
sockets. For, e.g., a CPU with two sockets, we divide the number of rows ({\it{n}}) by 2, and
allocate the memory for each relevant part of the matrix on its individual
socket. This ensures that each socket has (avg.) similar number of remote
accesses. For our experiments, this provided a boost of $\sim$5 -- 10\% to
performance, but we expect the optimizaton to be more beneficial with
increasing number of sockets.


\subsection{CX and PCA Implementation in Spark}

Spark provides a high-level programming model and execution engine for
fault-tolerant parallel and distributed computing, based on a core abstraction
called the \textit{resilient distributed dataset (RDD)}.  RDDs are immutable lazily
materialized distributed collections supporting functional programming
operations such as \texttt{map}, \texttt{filter}, and \texttt{reduce}, each of
which returns a new RDD.  RDDs may be loaded from a distributed file system,
computed from other RDDs, or created by parallelizing a collection created
within the user's application.  RDDs of key-value pairs may also be treated as
associative arrays, supporting operations such as \texttt{reduceByKey},
\texttt{join}, and \texttt{cogroup}.  Spark employs a lazy evaluation strategy
for efficiency.  Another major benefit of Spark over MapReduce is the use of
in-memory caching and storage so that data structures can be reused.

\subsection{Multi-node Spark Implementation}
\label{sec:cx_spark}
The main consideration when implementing CX and PCA is the efficient
implementation of operations involving the data matrix $A$.  Access to $A$ by
the PCA algorithm occurs through the \textsc{MultiplyGramian} and
\textsc{Multiply} routines with repeated invocations of
\textsc{MultiplyGramian} accounting for the majority of the execution time.
Access to $A$ by the CX algorithm occurs through the \textsc{RandomizedSVD}
routine, which in turn accesses $A$ only through the \textsc{MultiplyGramian}
and \textsc{Multiply} routines; similarly, repeated invocations of
\textsc{MultiplyGramian} account for the majority of the execution time.

The matrix $A$ is stored as an RDD containing one \texttt{IndexedRow} per row of the input matrix,
where each \texttt{IndexedRow} consists of the row's index and corresponding data vector.
This is a natural storage format for many datasets stored on a distributed or shared file
system, where each row of the matrix is formed from one record of the
input dataset, thereby preserving locality by not requiring data shuffling
during construction of $A$. In the case of the PCA algorithm, an initial pass is made over $A$ to compute
its column means, and a second pass is made to subtract out the means.

We compute \textsc{MultiplyGramian} in a form amenable to efficient distributed implementation
by exploiting the fact that the matrix product $A^TAB$ can be written as a sum of outer products,
as shown in Algorithm~\ref{alg:gram}. This allows for full parallelism across the rows of the matrix with
each row's contribution computed independently, followed by a summation step to accumulate the result.
This approach may be implemented in Spark as a \texttt{map} to form the outer products followed by a \texttt{reduce}
to accumulate the results:
\begin{verbatim}
def multiplyGramian(A: RowMatrix, B: LocalMatrix) =
  A.rows.map(row => row * row.t * B).reduce(_ + _)
\end{verbatim}
However, this approach forms $2m$ unnecessary temporary matrices of same dimension as the output matrix $n\times k$,
with one per row as the result of the \texttt{map} expression, and the \texttt{reduce} is not done in-place so it
too allocates a new matrix per row.
This results in high Garbage Collection (GC) pressure and makes poor use of the CPU cache, so
we instead remedy this by accumulating the results in-place by replacing the \texttt{map}
and \texttt{reduce} with a single \texttt{treeAggregate}.
The \texttt{treeAggregate} operation is equivalent to a map-reduce that executes in-place to accumulate the contribution of a
single worker node, followed by a
tree-structured reduction that efficiently aggregates the results from each worker.
The reduction is performed in multiple stages using a tree topology to avoid creating a single
bottleneck at the driver node to accumulate the results from each worker node.
Each worker emits a relatively large result with dimension $n\times k$, so the
communication latency savings of having multiple reducer tasks is significant.
\begin{verbatim}
def multiplyGramian(A: RowMatrix, B: LocalMatrix) = {
  A.rows.treeAggregate(LocalMatrix.zeros(n, k))(
    seqOp = (X, row) => X += row * row.t * B,
    combOp = (X, Y) => X += Y
  )
}
\end{verbatim}



