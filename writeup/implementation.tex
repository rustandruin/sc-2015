
\begin{figure*}[htp]
\includegraphics[scale=0.254]{images/jatin_a}
\includegraphics[scale=0.254]{images/jatin_b}
\caption{ Data access pattern for computing ${C}_{\it{i}}$ (left) and {\it{Res}}$_{\it{j}}$ (right) respectively.}
\label{fig:access_pattern}
\end{figure*}

\section{High Performance Implementation}
\label{sec:implementation}

We undertake two classes of high performance implementations for the CX method. 
We start with a highly optimized, close-to-the-metal C implementation
that focuses on obtaining peak efficiency from conventional multi-core
CPU chipsets and extend it to multiple nodes. 
%%%The purpose of this exercise is to establish a baseline for the CX method. 
Secondly, we implement the CX method in Spark, an emerging standard for parallel data analytics frameworks. 
%%%%%%%%%%%%%%%%%%%%%Implementing the method in Spark provides us with a scalable strategy to run on multiple nodes, and tackle a large scale dataset. 



\subsection {Single Node Implementation/Optimizations}
\label{sxn:single_node_opt}

    We now focus on optimizing the CX implementation on a single
    compute-node.
    %, aimed at exploiting the available SIMD units on each core, and the multiple cores across different sockets to speed up the performance. 
    We began by profiling our initial %%unoptimized 
    scalar serial CX
    code and optimizing the steps in the order of execution times.
    The most time is spent in computing sparse-sparse-dense matrix
    multiplication ($A^TAB$, Step 3, 90.6\%), followed by  sparse-dense matrix
    multiplication ($AQ$, Step 7, 9.1\%) and finally, QR
    decomposition (Step 4, 0.4\%)
    for a representative dataset that fits in main memory. These
    three kernels account for more than 99.9\% of
    the execution time.
    Recall that $A$ is a sparse matrix with dimensions
    $m \times n$
    and sparsity $s$, and $B$ is a dense $n \times k$ matrix.
    %%We first discuss the case of $A^TAB$.

\vspace*{0.05in} 
\subsubsection{Optimizing $\mathrm{Res}=A^TAB$}
    Optimizing sparse matrix-matrix  multiplication is an
    active area of research~\cite{ballard13,patwary15}; 
    state-of-the-art implementations
    are bound by the memory bandwidth and heavily
    underutilize the compute resources. 

    For our application, we exploit the following three observations:
    (1) One of the sparse matrices is the transpose of the other,   
    (2) One of the matrices is a dense matrix,   and    %%The sparse-sparse matrix multiplication is followed by a sparse-dense matrix multiplication. 
    (3) $n \gg k$ and $sm \gg k$.

    Exploiting associativity of multiplication, we compute $C$ = $AB$, 
    followed by $\mathrm{Res} = A^TC$. This reduces the run-time complexity from
    O({\it{n*(nsm)}}) to O({\it{k*(nsm)}}). Furthermore, we do not
    explicitly compute (or store) the transpose of $A$.  Consider the
    ${\it{i}}^{th}$ row, $A_i$. 
    By definition,  $C_i = A_i \cdot B$.
    The (${\it{j}},{\it{l}}$)$^{th}$ element of $\mathrm{Res}$,
    $\mathrm{Res}_{{\it{j}},{\it{l}}}$ =
    $\Sigma_{\it{p}}$($A^T_{{\it{j}},{\it{p}}}$ x $C$$_{{\it{p}},{\it{l}}}$) = 
    $\Sigma_{\it{p}}$($A_{{\it{p}},{\it{j}}}$ x
    $C$$_{{\it{p}},{\it{l}}}$).
%% 
%% 
%% 
%% 
    For {\it{p}} = {\it{i}}, this reduces to incrementing
    $\mathrm{Res}_{{\it{j}},{\it{l}}}$ by $A_{{\it{i}},{\it{j}}}$ x
    $C$$_{{\it{i}},{\it{l}}}$. 
%% 
%% 
%% 
    Thus, for each row {\it{i}}, 
    %having computed C$_{\it{i}}$, we can scale each entry by  $A_{{\it{i}},{\it{j}}}$ 
    having computed C$_{\it{i}}$, we
    increment $\mathrm{Res}_{{\it{j}},{\it{l}}}$ 
    by $A_{{\it{i}},{\it{j}}}$ x $C$$_{{\it{i}},{\it{l}}}$
    for {\it{j}}$\in$[1..{\it{n}}] and {\it{l}}$\in$[1..{\it{k}}].
     We now describe how we parallelize  to exploit
     data- and thread-level parallelism and other relevant
     optimizations.

     %%%%%%%%%%%%%%%%%%%%%%%%%%%%%%%%%%%%%%%%%%%%%%%%%%%%%%%%%%%%%%%%%%%%%%%%%%%%%%%%%%%%%%%%%%%%%%%%%%%%%%%%%%%%%%%%%%%%%




     %%%%%%%%%%%%%%%%%%%%%%%%%%%%%%%%%%%%%%%%%%%%%%%%%%%%%%%%%%%%%%%%%%%%%%%%%%%%%%%%%%%%%%%%%%%%%%%%%%%%%%%%%%%%%%%%%%%%%
     %%%%%%%%%%%%%%%%%%%%%%%%%%%%%%%%%%%%%%%%%%%%%%%%%%%%%%%%%%%%%%%%%%%%%%%%%%%%%%%%%%%%%%%%%%%%%%%%%%%%%%%%%%%%%%%%%%%%%
     \vspace*{0.05in}
     {\it{1. Exploiting SIMD}}: Refer to Figure~\ref{fig:access_pattern}. 
     Consider element $A_{{\it{i}},{\it{j}}}$. To compute
     $C_{\it{i}}$, we need to scale each element of
     $B_{{\it{j}}}$ by  $A_{{\it{i}},{\it{j}}}$ and add it to
     $C_{\it{i}}$ ({\it{j}}$\in$[1..{\it{n}}]) ($C_{\it{i}}$ +=
     $A_{{\it{i}},{\it{j}}}$ x $B_{{\it{j}}}$). Note that there are
     {\it{k}} elements in $B_{{\it{j}}}$, which are also stored
     consecutively (the matrix is stored in row-major form).
%%
On modern computing platforms, the SIMD width (number of simultaneous
     operations that can be performed) is
     growing~\cite{intel2}. SSE can perform 4
     single-precision floating point computations in a single op, while
     AVX %%(our SPARK platform) 
     performs 8 ops. Let $\mathcal{S}$ denote the SIMD width
     (defined as the number of double-precision floating point ops. per op
     -- which is half of the number of single-precision ops).
    %% 
     The pseudo-code~\footnote{Exact syntax varies with the ISA and
     compiler version.} for computing $C_{\it{i}}$ ($\forall$
     $A_{{\it{i}},{\it{j}}}$ $\neq$ 0) is
     \vspace*{0.03in}

     \hspace*{-0.13in}xmm\_a = {\it{vec\_load\_and\_splat}}($A_{{\it{i}},{\it{j}}}$); \\
     for\hspace*{0.02in}({\it{z}} = 0; {\it{z}} $<$ {\it{k}}; {\it{z}} += $\mathcal{S}$)\\
     \{\\
         \hspace*{0.2in}xmm\_c = {\it{vec\_load}}  ($C_{{\it{j}}}$ + z); \\
         \hspace*{0.2in}xmm\_b = {\it{vec\_load}}  ($B_{{\it{j}}}$ + z); \\
         \hspace*{0.2in}xmm\_ab = {\it{vec\_mul}}  (xmm\_a, xmm\_b); \\
         \hspace*{0.2in}xmm\_c = {\it{vec\_add}}  (xmm\_ab, xmm\_c); \\
         \hspace*{0.2in}{\it{vec\_store}} (xmm\_C,  $C_{{\it{j}}}$ + z); \\
     \}\\

     \vspace*{-0.05in}
     As is evident from the code, for each
     $A_{{\it{i}},{\it{j}}}$ $\neq$ 0 (the number of nonzeros ({\it{nnz}}) in  $A$), 
     we execute ($\lceil$$\frac{k}{\mathcal{S}}$$\rceil$) 
     {\it{add}} (and {\it{mul}}) operations, taking a total of
     $\lceil$$\frac{2*{\it{nnz}}*{\it{k}}}{\mathcal{S}}$$\rceil$ ops,
     a potential speedup of $\mathcal{S}$.  
     %%in terms of floating point operations executed.

     We now describe the vectorization of  $X =$ $A^TC$. %%, note that $C$ is a dense matrix. 
     As explained above, this
     requires incrementing X$_{{\it{j}}}$ 
     by $A_{{\it{i}},{\it{j}}}$ x $C$$_{{\it{i}}}$ 
     (both X$_{{\it{j}}}$ and $C$$_{{\it{i}}}$ have {\it{k}} elements each.)
     We use similar code to perform this computation at a cost of
    $\lceil$$\frac{2*{\it{nnz}}*{\it{k}}}{\mathcal{S}}$$\rceil$
    ops, a speedup of $\mathcal{S}$.
    %%in terms of floating point operations executed.

    \iffalse
    On some architectures, vector loads and stores are faster if
    memory addresses are 256-bit (or 512-bit aligned). 
    Since all our memory loads/stores start with each row of any
    matrix, we assign {\it{k}}  to be a multiple of 8, and align the
    starting addresses of all matrices to take advantage of such
    scenarios.
    \fi
    

     %%%%%%%%%%%%%%%%%%%%%%%%%%%%%%%%%%%%%%%%%%%%%%%%%%%%%%%%%%%%%%%%%%%%%%%%%%%%%%%%%%%%%%%%%%%%%%%%%%%%%%%%%%%%%%%%%%%%%
     %%%%%%%%%%%%%%%%%%%%%%%%%%%%%%%%%%%%%%%%%%%%%%%%%%%%%%%%%%%%%%%%%%%%%%%%%%%%%%%%%%%%%%%%%%%%%%%%%%%%%%%%%%%%%%%%%%%%%
     %%%%%%%%%%%%%%%%%%%%%%%%%%%%%%%%%%%%%%%%%%%%%%%%%%%%%%%%%%%%%%%%%%%%%%%%%%%%%%%%%%%%%%%%%%%%%%%%%%%%%%%%%%%%%%%%%%%%%

     \vspace*{0.05in}
     {\it{2. Exploiting multiple cores}}: As explained above, we
     decompose matrix multiplication into two steps: %each for each row of $C$$_{{\it{i}}}$, 
     we {\it{first}} compute
     $C$$_{{\it{i}}}$ ({\it{k}} elements), then update 
     Res$_{{\it{j}}}$ for each row {\it{j}}. The %%number of 
     executed 
     flops %%(and memory loads/stores) 
     is proportional to the number of
     non-zeros in the specific row of $A$. Thus a straightforward way
     to divide work %%between the the cores ($\mathcal{C}$ in total) 
     is to divide the rows %%between the cores
     such that each of the cores work on the same number of {\it{nnz's}}. 

     %However, this might result in some of the rows being split between cores.
     %For reasonable sized matrices, it suffices to assign a complete
     %row to a core, without any slowdown.

     Thus each core (or thread) 
     computes its starting and ending row
     index, and for each assigned row {\it{i}}, computes
     $C$$_{{\it{i}}}$. The next step is to update Res.
%%
     Two possibilites exist. One option is for each thread
     to maintain a local copy of $Res$, and reduce the results at the
     end.
     %%once all the threads are done executing, reduce the results. %% to form the final answer.
     However, even for moderately sized datasets, (e.g. $k$ = 32,
     {\it{n}} = 32K %%, and 8-bytes/element 
    $\sim$  
    4 MB/thread), the working set 
     {\it{exceeds}} the cache per core.
     %%Hence, with this approach, 
     %%For each assigned row,  would load and store the complete $Res$ matrix. 
     A more efficient approach is to maintain a
     single copy of $Res$ shared by the threads and updated using
     locks, as described next.
     %%executing on a single-node, locks are used as described next. 

     We initialize {\it{n}} locks, one for each row of the output
     matrix ($Res$).
     %%, and each thread grabs a lock, performs update to a row of the matrix, and releases the lock.
     Once an executing thread computes $C$$_{{\it{i}}}$,
     %%(as the first step of the matrix multiplication for an assigned row {\it{i}}), 
     for each  $A_{{\it{i}},{\it{j}}}$ $\neq$ 0, it grabs the
     ${\it{j}}^{th}$ lock, updates the row, and releases the lock. 
     %
     For realistic datasets, for sparsity({\it{s}})
     ($\sim$
     0.001 -- 0.005), there is a very low probability of two threads
     blocking on a lock ($\sim$1\% even with ${\mathcal{C}}$ =
     128). 
     %We show in the results section that the contention indeed is very low, and most of 
     %the parallelization overhead is due to the instruction overhead for grabbing and releasing the locks.

     %%%\Comment{Jatin}{What is the instruction overhead for grabbing alock?}

     %%%%%%%%%%%%%%%%%%%%%%%%%%%%%%%%%%%%%%%%%%%%%%%%%%%%%%%%%%%%%%%%%%%%%%%%%%%%%%%%%%%%%%%%%%%%%%%%%%%%%%%%%%%%%%%%%%%%%
     %%%%%%%%%%%%%%%%%%%%%%%%%%%%%%%%%%%%%%%%%%%%%%%%%%%%%%%%%%%%%%%%%%%%%%%%%%%%%%%%%%%%%%%%%%%%%%%%%%%%%%%%%%%%%%%%%%%%%
     %%%%%%%%%%%%%%%%%%%%%%%%%%%%%%%%%%%%%%%%%%%%%%%%%%%%%%%%%%%%%%%%%%%%%%%%%%%%%%%%%%%%%%%%%%%%%%%%%%%%%%%%%%%%%%%%%%%%%
     \vspace*{0.05in}
     {\it{3. Cache Blocking}}: For smaller values of {\it{n}}, our
     thread-level parallelization scheme scales near-linearly with
     increasing number of cores. However, for {\it{n}} $>$ 64K, we
     started noticing a drop in scaling. This is due to the working
     set growing larger than the size of the last-level cache, and
     thereby the computation becoming bound by the available memory
     bandwidth. In contrast, if most of the memory fetches can come
     from the caches, we can efficiently  utilize the floating
     point compute units on the node. We now
     describe the computation of the working set, and our algorithm
     for performing cache-friendly updates.

     During the execution of the algorithm, the matrix $B$ is
     accessed, which is shared between all the cores. Matrix $A$ is 
     streamed from the memory, so does not contribute to
     the working set. If each thread were to maintain its local copy
     of the $Res$ matrix, the total working set would occupy
     8{\it{kn}}*($\mathcal{T}$ + 1) bytes ($\mathcal{T}$ threads). For
     our system,  
     %%architecture, %%with $\mathcal{C}$ = 24, and matrix parameters of {\it{k}} = 32 and {\it{n}} = 128K, 
     the working set would be
     around 1 GB, which is too large to fit in the
     caches.\footnote{In this discussion, caches refers to the last
       level cache.} Instead,
   maintaining a shared copy of the $Res$ matrix reduces the working set size to
     8{\it{kn}} bytes, around 128 MB. Note that the
     total size is independent of the number of cores, and thus future
     proofs our implementation.
     %%%%with respect to increasing number of cores on a single node. 
     However, the working set size would still depend on the
     number of columns in $A$. We devise the
     following scheme to reduce it further to a given cache size.
     %of the computing platform..

     Instead of performing the computation for {\it{n}} columns, we
     divide it into chunks of {\it{n}}$'$ columns, such that
     2*8*{\it{k}}*{\it{n}}$'$ $\sim$ $\mathcal{C}$ (the cache size). Hence, with
     $\mathcal{C}$ = 15 MB,  {\it{n}}$'$$\sim$ 64K elements (we set
     {\it{n}} to be a multiple of {\it{n}}$'$). %%%for ease of implementation). 
     We thus perform the computation in
     $\lceil\frac{\it{n}}{\it{n}'}\rceil$ rounds, 
     updating the corresponding rows
     ([{\it{r}}$\lceil\frac{\it{n}}{\it{n}'}\rceil$..({\it{r}} +
     1)$\lceil\frac{\it{n}}{\it{n}'}\rceil$]
     in round {\it{r}}).
     Recall from the previous subsection that the number of flops
     executed per nonzero element in $A$ is
     $\lceil\frac{4{\it{k}}}{\mathcal{S}}\rceil$.
     Since the non zeros elements of $A$ are stored consecutively, 
     this may require loading each element
     $\lceil\frac{\it{n}}{\it{n}'}\rceil$ times. Hence, the flops/byte
     of the computation is around
     $\lceil\frac{4{\it{k}}}{\mathcal{S}}\rceil$/$\lceil\frac{\it{n}}{\it{n}'}\rceil$.
     Using our representative numbers, this is around 16 flops/byte,
     which is greater than the peak flops/byte of the platform (around
     10 flops/byte), and
     hence our application is not bound by memory bandwidth. With
     large values of {\it{n}}, we might end up being bandwidth bound
     -- in which case we need to modify the way $A$ is stored, by
     storing it in chunks of columns that would be accessed in each
     round. This format of representing $A$  helps 
     exploit the complete computational power of the processor, 
     %%keep the computation bound by the compute flops, 
     and only incurs a
     one-time cost for rearranging the elements of $A$.
     %{\it{n}}$'$ = 64K seems to be a resonable size for current architectures.

     %%%%%%%%%\Comment{Jatin}{How to store A might be an interesting way -- Say like n = 64K}.
     
     %%%%%%%%%%%%%%%%%%%%%%%%%%%%%%%%%%%%%%%%%%%%%%%%%%%%%%%%%%%%%%%%%%%%%%%%%%%%%%%%%%%%%%%%%%%%%%%%%%%%%%%%%%%%%%%%%%%%%
     %%%%%%%%%%%%%%%%%%%%%%%%%%%%%%%%%%%%%%%%%%%%%%%%%%%%%%%%%%%%%%%%%%%%%%%%%%%%%%%%%%%%%%%%%%%%%%%%%%%%%%%%%%%%%%%%%%%%%
     %%%%%%%%%%%%%%%%%%%%%%%%%%%%%%%%%%%%%%%%%%%%%%%%%%%%%%%%%%%%%%%%%%%%%%%%%%%%%%%%%%%%%%%%%%%%%%%%%%%%%%%%%%%%%%%%%%%%%

     \vspace*{0.05in}
     {\it{4. Multi-socket Optimization}}: 
     Multi-socket architectures are increasing being used,
     %for high-performance computing, 
     wherein each socket has its own
     compute and memory resources. 
     %%It is indeed possible for cores in any socket to access data present in the memory of the other sockets. However, 
     All cross-socket traffic goes through a
     cross-socket link, which has lower bandwidth than access to local
     DRAM/caches. Hence, we need to optimize for the amount of data
     transferred between sockets. %% to ensure optimal performance.

     %For our current application, %in order to reduce the inter-socket communication, 
     We divide the allocation of $Res$ equally between
     the sockets. For e.g., a CPU with 2 sockets, we divide the
     number of rows ({\it{n}}) by 2, and allocate the memory for each
     relevant part of the matrix on its individual socket. This
     ensures that %%(at an average), 
     each socket makes, on average, a similar number of
     remote accesses. For our experiments, this
     %%prescribed style of memory allocation 
     provided a boost of
     $\sim$5 -- 10\% to performance, but we expect the optimizaton
     to be more beneficial with increasing number of
     sockets.
     
     
     %%Current CPU dies have more than one socket~\cite{fds}.

     %%Given {it{k}}, and cache size $\mathcal{C}$, we desire 2 copies of the matrix to reside in cache -- hence, 
%%     As explained above, we decompose the matrix multiplication into two steps, %each for each row of $C$$_{{\it{i}}}$, 

%%%%%%%%%%%%%%     $A_{{\it{i}},{\it{j}}}$ $\neq$ 0 ({\it{nnz}} in total in matrix $A$), 

%%     https://software.intel.com/sites/landingpage/IntrinsicsGuide/


%%%%%%%     \vspace*{0.3in}


    
    %%As explained in Sec.~\ref{sec:5.1?}, 



    %ballard -- Communication Optimal Parallel Multiplication of Sparse Random Matrices
    %http://www.eecs.berkeley.edu/~odedsc/papers/spaa13-sparse.pdf

\subsubsection{Optimizing $AB$}

    This step refers to Step 7 in the algorithm description in
    Algorithm~\ref{alg:rsvd}. The data- and thread-level parallelization optimizations described 
    in the previous subsection (optimizing $A^TAB$) apply here, since
    there we explicitly  compute $C$ = $AB$. As far as cache blocking is
    concerned, since $C$ does not have to be memory resident, we now have
    to ensure that $B$ is completely cache resident (i.e. $8nk\le
    \mathcal{C}$). With increasing {\it{n}}, we again peform the
    computation in multiple rounds, with each round operating on 
    $n'=\frac{\mathcal{C}}{8k}$ rows of $B$.
%%
%%
%%%, and compute {\it{n}}$'$ (the number of columns of  accordingly. 
    Finally, as far as multi-socket optimizations are concerned, we
    divide the allocation of $C$, the output matrix in this case,
    between the various sockets, to reduce the amount of cross-socket
    memory traffic.





\iffalse
\begin{itemize}
\item Cache-Friendly 
\item SIMD
\item Thread- or core-level
\item Multi-socket 
\end{itemize}
\fi

\subsection {Multi-Node Implementation}
Consider $A^TAB$.
Similar to the multi-core implementation, we achieve load balancing by
dividing the rows %%between the computational nodes 
such that each node operates on the same number of nonzeros. 
%Since only the total number of non zeros in $A$ (and the number of rows) is known at the start of the CX computation, 
We perform this partitioning using a two step
process. In the first step, we equally divide the number of rows, 
%equally between the number of nodes, 
and each node reads in the corresponding
part of the matrix, and computes the number of non-zeros read. This is
followed by a redistribution step, where each node computes 
and distributes the relevant rows.
%so that each node has rows whose count of non zero elements is similar. 
The amount of data transferred between nodes is only a small fraction of the total input size
(measured to be $<$ 0.01\%), and this step is only performed once during the execution of
the algorithm.
%%%%%%%%%%%%%
Each node computes a local copy of $Res$, which is then
reduced globally to compute the final matrix. Note that $Res$ consists
of $n \times k$ elements, which occupies a few MBs even
for our largest 1 TB datset  (recall $m \gg n$). 

As far as $QR$ decomposition is concerned, given the small size of the
matrix ($n \times k$), it is performed on a
single-node, but parallelized to exploit the multiple cores, with the resultant matrix being broadcast
to all other nodes at the end of the computation.
%%%%There is an explicit barrier at the end of each of the {\it{q}} iterations of Algorithm 1.
%%
%%
%%
A similar work division scheme is used to compute $AB$ (Step 7) in a
distributed fashion. The final two steps (the ThinSVD and the small matrix
multiplication) are performed on a single-node.
%%%
%%In practice, we achieved near-linear scaling for our test systems with less than 50 nodes.



\subsection{CX and PCA Implementation in Spark}

Spark provides a high-level programming model and execution engine for
fault-tolerant parallel and distributed computing, based on a core abstraction
called the \textit{resilient distributed dataset (RDD)}.  RDDs are immutable lazily
materialized distributed collections supporting functional programming
operations such as \texttt{map}, \texttt{filter}, and \texttt{reduce}, each of
which returns a new RDD.  RDDs may be loaded from a distributed file system,
computed from other RDDs, or created by parallelizing a collection created
within the user's application.  RDDs of key-value pairs may also be treated as
associative arrays, supporting operations such as \texttt{reduceByKey},
\texttt{join}, and \texttt{cogroup}.  Spark employs a lazy evaluation strategy
for efficiency.  Another major benefit of Spark over MapReduce is the use of
in-memory caching and storage so that data structures can be reused.

\subsection{Multi-node Spark Implementation}
\label{sec:cx_spark}
The main consideration when implementing CX and PCA is the efficient
implementation of operations involving the data matrix $A$.  Access to $A$ by
the PCA algorithm occurs through the \textsc{MultiplyGramian} and
\textsc{Multiply} routines with repeated invocations of
\textsc{MultiplyGramian} accounting for the majority of the execution time.
Access to $A$ by the CX algorithm occurs through the \textsc{RandomizedSVD}
routine, which in turn accesses $A$ only through the \textsc{MultiplyGramian}
and \textsc{Multiply} routines; similarly, repeated invocations of
\textsc{MultiplyGramian} account for the majority of the execution time.

The matrix $A$ is stored as an RDD containing one \texttt{IndexedRow} per row of the input matrix,
where each \texttt{IndexedRow} consists of the row's index and corresponding data vector.
This is a natural storage format for many datasets stored on a distributed or shared file
system, where each row of the matrix is formed from one record of the
input dataset, thereby preserving locality by not requiring data shuffling
during construction of $A$. In the case of the PCA algorithm, an initial pass is made over $A$ to compute
its column means, and a second pass is made to subtract out the means.

We compute \textsc{MultiplyGramian} in a form amenable to efficient distributed implementation
by exploiting the fact that the matrix product $A^TAB$ can be written as a sum of outer products,
as shown in Algorithm~\ref{alg:gram}. This allows for full parallelism across the rows of the matrix with
each row's contribution computed independently, followed by a summation step to accumulate the result.
This approach may be implemented in Spark as a \texttt{map} to form the outer products followed by a \texttt{reduce}
to accumulate the results:
\begin{verbatim}
def multiplyGramian(A: RowMatrix, B: LocalMatrix) =
  A.rows.map(row => row * row.t * B).reduce(_ + _)
\end{verbatim}
However, this approach forms $2m$ unnecessary temporary matrices of same dimension as the output matrix $n\times k$,
with one per row as the result of the \texttt{map} expression, and the \texttt{reduce} is not done in-place so it
too allocates a new matrix per row.
This results in high Garbage Collection (GC) pressure and makes poor use of the CPU cache, so
we instead remedy this by accumulating the results in-place by replacing the \texttt{map}
and \texttt{reduce} with a single \texttt{treeAggregate}.
The \texttt{treeAggregate} operation is equivalent to a map-reduce that executes in-place to accumulate the contribution of a
single worker node, followed by a
tree-structured reduction that efficiently aggregates the results from each worker.
The reduction is performed in multiple stages using a tree topology to avoid creating a single
bottleneck at the driver node to accumulate the results from each worker node.
Each worker emits a relatively large result with dimension $n\times k$, so the
communication latency savings of having multiple reducer tasks is significant.
\begin{verbatim}
def multiplyGramian(A: RowMatrix, B: LocalMatrix) = {
  A.rows.treeAggregate(LocalMatrix.zeros(n, k))(
    seqOp = (X, row) => X += row * row.t * B,
    combOp = (X, Y) => X += Y
  )
}
\end{verbatim}



