We investigate the performance, scalability, and applicability of low-rank
matrix approximation algorithms, including PCA and randomized CX low-rank
matrix factorizations, on a sparse 1TB mass spectrometry imaging (MSI) dataset
and a dense 2TB ocean temperature dataset, using the Apache Spark high-level
cluster computing framework on an Amazon~EC2 cluster, a Cray~XC40 system, and
an experimental Cray cluster. Scientific applications such as MSI data
analysis and Empirical Orthogonal Function decomposition of climate fields
provide a very different use case for Spark than is provided by typical commercial
workloads.

We obtained similar performance on all three platforms
for the MSI application; using Spark we were able to process the 1TB size
dataset in under 30 minutes with 960 cores on all systems, with the fastest
times obtained on the experimental Cray cluster; similarly, for the climate
science application, we were able to process the 2TB size dataset in under 30
minutes with 960 cores on the Amazon~EC2 cluster. In comparison, the C
implementation of our CX decomposition was 21X faster (on an Amazon EC2
system), due to careful cache optimizations, bandwidth-friendly access of
matrices and vector computation using SIMD units. 
We report these results along with scientific insights gleaned from the
MSI and climate datasets, and conclude with broader implications on the
hardware and software issues arising in supporting data-centric workloads in
parallel and distributed environments.
